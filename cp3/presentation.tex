\documentclass{beamer}

\usepackage[T1,T2A]{fontenc}
\usepackage[utf8]{inputenc}
\usepackage[russian, english]{babel}

\usepackage{amsmath, amsfonts, amsthm, amscd}
%\usepackage{graphics}
\usepackage{graphicx}
\usepackage[matrix,arrow,curve]{xy}
%\usepackage{multicol}

%----------------------------------------------------

\newtheorem{problemR}{Проблема}
\newtheorem{aim}{Цель}


\newtheorem{construction}{Construction}
\newtheorem{problems}{Problems}
\newtheorem{conjecture}{Conjecture}
\newtheorem{question}{Question}

\mode<presentation>
\usetheme{Madrid}

\title{}
\date{2021}
\author{Константин Амеличев}


\begin{document}



\begin{frame}
\begin{center}


{\large \scshape

\bigskip

\bigskip

3d-Renderer с нуля

\bigskip
}



Амеличев Константин, ПМИ 191\\

\bigskip

\bigskip
\bigskip
\bigskip
\bigskip

\end{center}
\end{frame}

\begin{frame}

\frametitle{Цель работы}

Разработка библиотеки для отрисовки объектов из 3d-пространства

\bigskip

Из пререквизитов~--- только отрисовка пикселей на экране.

\end{frame}

\begin{frame}

\frametitle{Поставленные задачи}

\begin{itemize}
\item Изучение теории
\item Сборка пайплайна
\item Создание приложения
\item Тестирование
\item Документация
\end{itemize}

\end{frame}


\begin{frame}
\frametitle{Pipeline}

\begin{itemize}
\item Перенос объекта в систему координат камеры.
\item Клиппинг (выбор объектов для отображения).
\item Проективное преобразование
\item Растеризация
\end{itemize}

\end{frame}

\begin{frame}

\frametitle{Изучение теории}

\begin{itemize}

\item Однородные координаты для работы с 3d-пространством
\begin{itemize}
\item $\begin{pmatrix} x \\ y \\ z \\ w \end{pmatrix} \iff \begin{pmatrix} \frac{x}{w} \\ \frac{y}{w} \\ \frac{z}{w} \\ 1 \end{pmatrix} \iff \begin{pmatrix} x \\ y \\ z \end{pmatrix}$
\end{itemize}
\item Повороты и сдвиги
\item Проективное преобразование
\item Работа с матрицами

\end{itemize}


\end{frame}

\begin{frame}
\frametitle{Алгоритмическая часть}

\begin{itemize}
\item Клиппинг
% Алгоритм Уайлера — Атертона
\item Растеризация
\begin{itemize}
	\item Выбор пикселей треугольника
	\item Вычисление z-value
\end{itemize}
\end{itemize}

\end{frame}


\begin{frame}
\frametitle{Программная часть}

\begin{itemize}
\item Renderer
\item Camera
\item World
\item Window (SFML)
\end{itemize}

\end{frame}

% \begin{frame}
% \frametitle{Схема взаимодействия}

% \end{frame}

\begin{frame}
\frametitle{Программная часть}

\begin{itemize}
\item Application
\item Doxygen
\item CxxTest + Github Actions
\end{itemize}
\end{frame}

\begin{frame}
\frametitle{Результаты}

КАРТИНКА

\end{frame}


\end{document}
