% Это основная команда, с которой начинается любой \LaTeX-файл. Она отвечает за тип документа, с которым связаны основные правил оформления текста.
\documentclass{article}

% Здесь идет преамбула документа, тут пишутся команды, которые настраивают LaTeX окружение, подключаете внешние пакеты, определяете свои команды и окружения. В данном случае я это делаю в отдельных файлах, а тут подключаю эти файлы.

% Здесь я подключаю разные стилевые пакеты. Например возможности набирать особые символы или возможность компилировать русский текст. Подробное описание внутри.
\usepackage{packages}

% Здесь я определяю разные окружения, например, теоремы, определения, замечания и так далее. У этих окружений разные стили оформления, кроме того, эти окружения могут быть нумерованными или нет. Все подробно объяснено внутри.
\usepackage{environments}

% Здесь я определяю разные команды, которых нет в LaTeX, но мне нужны, например, команда \tr для обозначения следа матрицы. Или я переопределяю LaTeX команды, которые работают не так, как мне хотелось бы. Типичный пример мнимая и вещественная часть комплексного числа \Im, \Re. В оригинале они выглядят не так, как мы привыкли. Кроме того, \Im еще используется и для обозначения образа линейного отображения. Подробнее описано внутри.
\usepackage{commands}

% Пакет для титульника проекта
\usepackage{titlepage}

% Здесь задаем параметры титульной страницы
\setUDK{192.168.1.1}
% Выбрать одно из двух
\setToResearch
%\setToProgram

\setTitle{Базисы шмазисы}
\setStage{(промежуточный, этап 1)}
\setGroup{201}
%сюда можно воткнуть картинку подписи
\setStudentSgn{подпись студента}
\setStudent{И.И.Иванов}
\setStudentDate{11.06.2021}
\setAdvisor{Дмитрий Витальевич Трушин}
\setAdvisorTitle{доцент, к.ф.-м.н.}
\setAdvisorAffiliation{ФКН НИУ ВШЭ}
\setAdvisorDate{12.06}
\setGrade{11}
%сюда можно воткнуть картинку подписи
\setAdvisorSgn{подпись руководителя}
\setYear{2021}


% С этого момента начинается текст документа
\begin{document}

% Эта команда создает титульную страницу
\makeTitlePage

% Здесь будет автоматически генерироваться содержание документа
\tableofcontents

% Данное окружение оформляет аннотацию: краткое описание текста выделенным абзацем после заголовка
\begin{abstract}
Текст аннотации. Здесь кратко в два-три предложения описываем, что происходит в работе.
\end{abstract}


\section{Введение}

Во введении надо кратко описать область, в которой будет ваша работа, потом рассказать о поставленной задаче, далее о том, что вы будете делать.


Со временем, когда вы начнете набирать содержательную часть, вы будете модифицировать введения, меняя фразы о своих планах, на описание проделанной работы. Так же в конце введения обычно принято писать обзор структуры содержательной части, чтобы можно было сориентироваться в происходящем, не начиная читать содержательную часть.

Здесь вы пишите какие в каких источниках какая информация находится. Это нужно для того, чтобы читающий мог понять, где искать информацию в нужном направлении после изучения вашего текста. Этот пункт можно не выделять отдельно. Его можно слить с другими, например сослаться на источники во время описания работы или описания структуры курсовой.

Пример цитирования источника. Если в тексте не будет ни одного цитирования, то BiB-TeX выдаст ошибку при генерации списка литературы. В списке будут только те источники, на которые есть ссылки.

Note also that very recently several constructions of~\cite{Elkik73} were clarified and simplified by Gabber and Ramero in~\cite[Chapter~5]{GabRam}. Some other source with url~\cite{GGO}

\section{Описание функциональных и нефункциональных требований к программному проекту}

Этот пункт нужен только для программных проектов. В нем вы описываете, что у вас вообще должна быть за программа. На каком языке вы ее пишите. Что она должна делать. Подробное описание. Например, у вас пишется библиотека для работы с многочленами. Она должна предоставлять такие-то классы, пример использования. Она предоставляет такие-то методы, пример использования. Такая-то сложность методов. Например, еще можно написать, что вы тестируете библиотеку с помощью консольного приложения, данные считываются так-то, тестируются такие-то вещи, такие-то вещи выводятся. И так далее и тому подобное.

\section{Содержательная часть}

Здесь идет планомерное изложение информации от начала до конца. Тут не нужна никакая философия или объяснения, все это было во введении. Тут сухой математический текст с определениями, формулировками и где надо доказательствами. Содержательную часть можно бить на части, чтобы структурировать изложение.

\subsection{Содержательная часть 1}

\subsection{Содержательная часть 2}

% Здесь автоматически генерируется библиография. Первая команда задает стиль оформления библиографии, а вторая указывает на имя файла с расширением bib, в котором находится информация об источниках.
\bibliographystyle{plainurl}
\bibliography{bibl}



% С этого момента глобальная нумерация идет буквами. Этот раздел я добавил лишь для демонстрации возможностей LaTeX, его можно и нужно удалить и он не нужен для курсового проекта непосредственно.
\appendix

Проведем небольшой обзор возможностей \LaTeX. Далее идет обзорный кусок, который надо будет вырезать. Он приведен лишь для демонстрации возможностей \LaTeX.

\section{Нумеруемый заголовок}
Текст раздела
\subsection{Нумеруемый подзаголовок}
Текст подраздела
\subsubsection{Нумеруемый подподзаголовок}
Текст подподраздела

\section*{Не нумеруемый заголовок}
Текст раздела
\subsection*{Не нумеруемый подзаголовок}
Текст подраздела
\subsubsection*{Не нумеруемый подподзаголовок}
Текст подподраздела


\paragraph{Заголовок абзаца} Текст абзаца

Формулы в тексте набирают так $x = e^{\pi i}\sqrt{\text{формула}}$. Выключенные не нумерованные формулы набираются либо так:
\[
x = e^{\pi i}\sqrt{\text{формула}}
\]
Либо так
$$
x = e^{\pi i}\sqrt{\text{формула}}
$$
Первый способ предпочтительнее при подаче статей в журналы AMS, потому рекомендую привыкать к нему.

Выключенные нумерованные формулы:
\begin{equation}\label{Equation1}
% \label{имя-метки} эта команда ставит метку, на которую потом можно сослаться с помощью \ref{имя-метки}. Метки можно ставить на все объекты, у которых есть автоматические счетчики (номера разделов, подразделов, теорем, лемм, формул и т.д.
x = e^{\pi i}\sqrt{\text{формула}}
\end{equation}
Или не нумерованная версия
\begin{equation*}
x = e^{\pi i}\sqrt{\text{формула}}
\end{equation*}

Уравнение~\ref{Equation1} радостно занумеровано.

Лесенка для длинных формул
\begin{multline}
x = e^{\pi i}\sqrt{\text{очень очень очень длинная формула}}=\\
\tr A - \sin(\text{еще одна очень очень длинная формула})=\\
\cos z \Im \varphi(\text{и последняя длинная при длинная формула})
\end{multline}

Многострочная формула с центровкой
\begin{gather}
x = e^{\pi i}\sqrt{\text{очень очень очень длинная формула}}=\\
\tr A - \sin(\text{еще одна очень очень длинная формула})=\\
\cos z \Im \varphi(\text{и последняя длинная при длинная формула})
\end{gather}

Многострочная формула с ручным выравниванием. Выравнивание идет по знаку $\&$, который на печать не выводится.
\begin{align}
x = &e^{\pi i}\sqrt{\text{очень очень очень длинная формула}}=\\
&\tr A - \sin(\text{еще одна очень очень длинная формула})=\\
&\cos z \Im \varphi(\text{и последняя длинная при длинная формула})
\end{align}

\begin{theorem}
Текст теоремы
\end{theorem}
\begin{proof}
В специальном окружении оформляется доказательство.
\end{proof}

\begin{theorem}[Имя теоремы]
Текст теоремы
\end{theorem}
\begin{proof}[Доказательство нашей теоремы]
В специальном окружении оформляется доказательство.
\end{proof}

\begin{definition}
Текст определения
\end{definition}

\begin{remark}
Текст замечания
\end{remark}

\paragraph{Перечни:} Нумерованные
\begin{enumerate}
\item Первый
\item Второй
\begin{enumerate}
\item Вложенный первый
\item Вложенный второй
\end{enumerate}
\end{enumerate}

Не нумерованные

\begin{itemize}
\item Первый
\item Второй
\begin{itemize}
\item Вложенный первый
\item Вложенный второй
\end{itemize}
\end{itemize}


% Здесь текст документа заканчивается
\end{document}
% Начиная с этого момента весь текст LaTeX игнорирует, можете вставлять любую абракадабру.
